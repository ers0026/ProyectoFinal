\capitulo{7}{Conclusiones y Líneas de trabajo futuras}

\section{Conclusiones}

Tras haber finalizado con la realización del proyecto, podemos decir que hemos cumplido la mayoría de los objetivos y los requisitos que se propusieron al iniciar el mismo.

El objetivo principal, era la creación de contratos inteligentes en la red Ethereum o las redes privadas que ofrece esta misma, mediante una página web, en un principio hemos trabajado todo mediante la red Localhost, con la posibilidad de en un futuro crear un dominio y que fuera de ámbito general.

Gracias a la investigación y realización de este proyecto, he aprendido a usar nuevos lenguajes de programación (Solidity, React, Ethereum) y a coger experiencia en otros que ya había visto en la carrera pero con apenas experiencia en ellos (HTML, CSS, PHP). 

En otro caso, gracias a la colaboración de HP en este proyecto de investigación, he aprendido el funcionamiento de una empresa. También nos hemos encontrado con limitaciones en el desarrollo del proyecto, ya que al no estar tan desarrollada la tecnología sobre blockchain, la información no era tan abundante en las redes como en otros lenguajes de programación.

Como apunte final, cabe destacar que los problemas o errores encontrados durante la realización del proyecto, me han ayudado a investigar e intentar buscar soluciones de forma más o menos efectiva dependiendo del error. Este proyecto ha sido un proceso de auto aprendizaje contando con la ayuda de Pablo Tejedor (tutor de HP) y de Ángel Arroyo (Tutor académico).  

\section{Líneas de trabajo futuras}

Posibilidades de expansión

\begin{itemize}
	\item Ampliación del catalogo CNAE para dar mayor especificación al producto.
	\item Ampliación de la aplicación para no usar solo la red ethereum sino también otras blockchain públicas o privadas.
	\item Creación de un sistema de usuarios ampliado (diferentes tipos de empresa según su sector de actividad).
	\item Mejoras visuales de la aplicación.
	\begin{enumerate}[a)]
		\item Tracking del producto (incorporación de servicio de mapas tipo Google Maps a la app). usar marcado GPS.
		\item Añadir fotografía del producto añadido.
		\item Anexar documentación oficial sobre el producto y introducirlo en la blockchain.
	\end{enumerate}
	\item Publicar la página en servicio en linea, así la pagina estará siempre disponible para su uso.
\end{itemize}


